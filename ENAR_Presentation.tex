\documentclass[pdf]{beamer}
\usetheme{Boadilla}
\usecolortheme{beaver}

\usepackage{caption} %smaller caption setup
\captionsetup[figure]{font=scriptsize, labelfont= scriptsize} 
\usepackage{amssymb}
\usepackage{amsmath}
\usepackage{graphicx}
\usepackage{float}
\usepackage{color,xcolor,ucs}

\mode<presentation>{}

% shorter title in the bottom
\title[Multiple Imputation]{Multiple Imputation of an Expensive Covariate in Outcome Dependent Sampling Designs for Longitudinal Data}
\author[Chiara Di Gravo]{Chiara Di Gravio, Ran Tao, Jonathan S Schildcrout}
\institute[]{Vanderbilt University \\ Virtual ENAR, 2020}



\date{March 25, 2020}


\begin{document}
	
\begin{frame}
\titlepage
\end{frame}	

\begin{frame}
\frametitle{Outline}
\tableofcontents
\end{frame}	


%------------------------------------------------

\section{Motivation}

\begin{frame}
\frametitle{Motivation}
\begin{itemize}
	\setlength\itemsep{1.5em}
	\item In longitudinal studies when exposure ascertainment costs limit sample size, it is desirable to target a sample of informative subjects
	\item Different methods have been developed to efficiently select patients for exposure ascertainment
	\item Analysis is usually done using only subjects in whom exposure was collected, or combining partial data on those not sampled and complete data on those sampled using full likelihood approaches or multiple imputation
\end{itemize}
\end{frame}

\begin{frame}
\frametitle{Motivation}
\begin{itemize}
	\setlength\itemsep{1.5em}
	\item For today, we focus on multiple imputation (MI)
	\item MI could be more \textbf{efficient} than conditional likelihood analysis, and often \textbf{easier to implement} than full likelihood approaches
\end{itemize}
\end{frame}	

\begin{frame}
\frametitle{Lung Health Study}
\begin{itemize}
	\setlength\itemsep{0.8em}
	\item Lung Health Study (LHS) data, a multi-center RCT of smokers with mild chronic obstructive pulmonary disease
	\item We focus on a single SNP found to be a modifier of lung function decline. This is the expensive exposure.
	\item We are interested in the association between SNP and $FEV\%$, and how the association changes over time
	\item We consider a scenario in which phenotype and covariate data are available on all subjects but resource constraints only permit SNP to be collected in $20\%$ of the subjects.
\end{itemize}
\end{frame}	

\begin{frame}
\frametitle{Lung Health Study}
\begin{itemize}
	\setlength\itemsep{0.8em}
	\item The mixed effects model used for our analyses is:
	\begin{equation}\nonumber
	Y_{ij} = \beta_0 + \beta_s SNP_i + \beta_t t_{ij} + \beta_{st}SNP_it_{ij} + \beta_c c_{i}+ b_{0i} + b_{1i}t_{ij} + \epsilon_{ij}
	\end{equation}
	\item $Y_{ij}$ is $FEV\%$ for subject $i$ at visit $j$
	\item $snp_i$ is an indicator for the presence of at least one copy of the allele at rs177852
	\item $t_{ij}$ is the time variable
	\item $(b_{0i}, b_{1i})$ are the random intercept and slope for subject $i$
	\item $c_i$ is a continuous baseline covariate
	\item $\epsilon_{ij}$ is assumed to be normally distributed and independent of the random effects
\end{itemize}
\end{frame}	

%-------------------------------------------

\section{Outcome Dependent Sampling}

\begin{frame}
\frametitle{Outcome Dependent Sampling}
\begin{itemize}
	\setlength\itemsep{0.8em}
	\item Longitudinal outcome data and basic covariate data ($\boldsymbol{Y}_i$, $\boldsymbol{T}_i$, $\boldsymbol{C}_i$) are available, but resource constraints allow us to collect SNP on $20\%$ of the subjects
	\item We want to select the most informative individuals using outcome dependent sampling (ODS)
	\item Sampling is based on strata defined by low-dimensional summaries of $\boldsymbol{Y}_i$:
	\begin{itemize}
		\item $E(Y_{ij}) = q_{0i} + q_{1i}t_{ij}$
		\item $q_{0i}$ is the subject-specific mean of $FEV\%$ at baseline and $q_{1i}$ is the subject-specific rate of change
		\item Sort values of $q_{0i}$ and/or $q_{1i}$ and introduce cut-points that define sampling strata from which we sample with different probabilities 
		\end{itemize}
\end{itemize}
\end{frame}

\begin{frame}
\frametitle{Different Sampling Scheme}
\begin{itemize}
	\setlength\itemsep{0.8em}
	\item \textbf{Random sampling}
	\item \textbf{ODS: intercept sampling and slope sampling} 
	\begin{figure}
		\centering
		\includegraphics[width=0.9\textwidth]{Designs.png}
	\end{figure}
\end{itemize}
\end{frame}

%-------------------------------------------

\section{Multiple Imputation (MI)}

\begin{frame}
\frametitle{MI Background}
\begin{itemize}
	\setlength\itemsep{0.8em}
	\item ODS allows us to select the most informative individuals for whom $SNP_i$ will be collected, and to increase the estimates efficiency
	\item In many circumstances, we can improve estimates’ efficiency by using all available data and imputing $SNP_{i}$ in those who were not sampled ($S_i = 0$)
	\item Because sampling only depends on $\boldsymbol{X}_{oi}$ and $\boldsymbol{Y}_i$:
	\begin{equation}\nonumber
	\resizebox{.9 \textwidth}{!} 
	{
	$pr(SNP_{i} | \boldsymbol{X}_{oi}, \boldsymbol{Y}_{i}, S_i = 0) = pr(SNP_{i} | \boldsymbol{X}_{oi}, \boldsymbol{Y}_{i})
	= pr(SNP_{i} | \boldsymbol{X}_{oi}, \boldsymbol{Y}_{i}, S_i = 1)$
	}
	\end{equation}
	\item Multiple imputation should provide unbiased and valid estimates 
\end{itemize}
\end{frame}

\begin{frame}
\frametitle{Imputation Model}
\begin{itemize}
	\setlength\itemsep{0.8em}
	\item We construct the imputation model in a straightforward way using available data on subjects. By Bayes' theorem:
	\begin{equation}\nonumber
	\resizebox{.95 \textwidth}{!} 
	{
	$\frac{pr(SNP_{i} = 1 | \boldsymbol{X}_{oi}, \boldsymbol{Y}_i, S_i = 0)}{pr(SNP_{i} = 0 | \boldsymbol{X}_{oi}, \boldsymbol{Y}_i, S_i = 0)} = \frac{f(\boldsymbol{Y}_i | SNP_{i}  = 1, \boldsymbol{X}_{oi}, S_i = 1)}{f(\boldsymbol{Y}_i |SNP_{i} = 0, \boldsymbol{X}_{oi}, S_i = 1)}
	\frac{pr(SNP_{i} = 1 | \boldsymbol{X}_{oi}, S_i = 1)}{pr(SNP_{i} = 0 | \boldsymbol{X}_{oi},S_i = 1)}$
	}
	\end{equation}

	\item We assume the Gaussian linear mixed model 
	\item After log-transforming both sides of the equations and doing some algebra, the imputation model is:
	\begin{equation}\nonumber
	\resizebox{.9 \textwidth}{!} 
	{
	$\boldsymbol{Y^{T}_i V_i^{-1}(\mu_{1,i} - \mu_{0,i}) - \frac{1}{2}(\mu_{1,i}^TV_i^{-1}\mu_{1,i} - \mu_{0,i}^TV_i^{-1}\mu_{0,i})}
	+ log\left(\frac{P(SNP_{i} = 1 | \boldsymbol{X}_{oi})}{P(SNP_{i} = 0 | \boldsymbol{X}_{oi})}\right)$
	}
	\end{equation}
	where $\boldsymbol{\mu_{x,i}} = E[\boldsymbol{Y}_i | SNP_{i} = x, \boldsymbol{X_{oi}}]$ and $\boldsymbol{V}_i = Cov(\boldsymbol{Y}_i | SNP_{i}, \boldsymbol{X}_{oi})$
\end{itemize}
\end{frame}

%-------------------------------------------

\section{Simulation Study}

\begin{frame}
\frametitle{Simulation Settings}
\begin{itemize}
	\item We simulate data based on the Lung Health Study
	\item We consider a cohort of 2,000 subjects from which we sampled 400
	\begin{equation}\nonumber
	Y_{ij} = \beta_0 + \beta_s SNP_i + \beta_t t_{ij} + \beta_{st}SNP_it_{ij} + \beta_c c_{i} + b_{0i} + b_{1i}t_{ij} + \epsilon_{ij}
	\end{equation}
	\begin{itemize}
		\setlength\itemsep{0.5em}
		\item $(\beta_0, \beta_s, \beta_t, \beta_{st}, \beta_{c}) = (75, -0.5, -1, -0.5, -2)$
		\item $(b_{0i}, b_{1i}) \sim N(\boldsymbol{0}, \boldsymbol{D})$ 
		\item $\sigma^2_{b0} = 81, \sigma^2_{b1} = 1.56$, $\sigma_{b0, b1} = 0$
		\item $\epsilon_{ij} \sim N(0, \sigma_e^2 = 12.25)$
	\end{itemize}
	\item We consider 3 different sampling designs
	\item We consider balanced and complete data, balanced and incomplete data, unbalanced data
\end{itemize}
\end{frame}

\begin{frame}
\frametitle{Case 1: Balanced and Complete Data}
\begin{itemize}
	\setlength\itemsep{0.8em}
	\item We impute $SNP$ using the collected information on $Y_{ij}$ at all time points
	\item Imputation model:
	\begin{equation}\nonumber
	\resizebox{.9 \textwidth}{!} 
	{
	$log\left(\frac{SNP_i = 1 | c_i, \boldsymbol{Y}_i}{SNP_i = 0 | c_i, \boldsymbol{Y}_i}\right) = \gamma_0 + \gamma_1 c_i + \gamma_2 y_{i1} + \gamma_3 y_{i2} + \gamma_4 y_{i3} + \gamma_5 y_{i4} + \gamma_6 y_{i5}$
	}
	\end{equation}
	\item If we use M = 50 imputations, coefficients' estimates and standard errors:
	\end{itemize}
 \begin{table}[H]
 	\centering
 	\resizebox{0.95\textwidth}{!}{
	\begin{tabular}{c|ccccc}
		\hline
		Sampling & $\beta_0$ & $\beta_t$ & $\beta_s$ & $\beta_{st}$ & $\beta_c$ \\
		\hline
		\emph{Random} & 75.00 (0.38) & -1.00 (0.05) & -0.49 (1.00) & -0.49 (0.14) & -2.00 (0.22) \\
		\emph{Intercept} & 75.00 (0.29) & -1.00 (0.05) & -0.51 (0.66) & -0.49 (0.14) & -2.01 (0.21) \\
		\emph{Slope} & 75.00 (0.38) & -1.00 (0.04) & -0.49 (1.02) & -0.50 (0.09) & -2.00 (0.23) \\
		\hline
		\textbf{Truth} & 75.00 & -1.00 & -0.50 & -0.50 & -2.00 \\
		\hline
	\end{tabular}
	}
\end{table}
\end{frame}

\begin{frame}
\frametitle{Case 2: Balanced and Incomplete Data}
\begin{itemize}
	\setlength\itemsep{0.8em}
	\item Since not every subject is measured at all visits we cannot use the same model as in the complete data case.
	\item What about using the mean of $Y_{ij}$?
		\begin{equation}\nonumber
		log\left(\frac{snp_i = 1 | c_i, \boldsymbol{y}_i}{snp_i = 0 | c_i, \boldsymbol{y}_i}\right) = \gamma_0 + \gamma_1 c_i + \gamma_2 \bar{y}_{i}
		\end{equation}
	\item If we use M = 50 imputations, coefficients' estimate and standard errors:
\end{itemize}
\begin{table}[H]
\centering
\resizebox{0.95\textwidth}{!}{
	\begin{tabular}{c|ccccc}
		\hline
		Sampling & $\beta_0$ & $\beta_t$ & $\beta_s$ & $\beta_{st}$ & $\beta_c$ \\
		\hline
		\emph{Random} & 75.25 (0.37) & -1.10 (0.04) & -1.32 (0.98) & -0.18 (0.09) & -1.93 (0.22) \\
		\emph{Intercept} & 75.07 (0.36) & -1.10 (0.04) & -0.49 (0.87) & -0.16 (0.09) & -2.00 (0.21)\\
		\emph{Slope} & 75.53 (0.33) & -1.05 (0.04) & -0.75 (0.84) & -0.33 (0.10) & -1.85 (0.22) \\
		\hline
		\textbf{Truth} & 75.00 & -1.00 & -0.50 & -0.50 & -2.00 \\
		\hline
	\end{tabular}
}
\end{table}
\end{frame}

\begin{frame}
\frametitle{Case 2: Balanced and Incomplete Data (cont'd)}
\begin{itemize}
	\item Imputation Model
	\begin{equation}\nonumber
	\resizebox{.9 \textwidth}{!} 
	{
		$\boldsymbol{Y^{T}_i V_i^{-1}(\mu_{1,i} - \mu_{0,i}) - \frac{1}{2}(\mu_{1,i}^TV_i^{-1}\mu_{1,i} - \mu_{0,i}^TV_i^{-1}\mu_{0,i})}
		+ log\left(\frac{P(SNP_{i} = 1 | X_{oi})}{P(SNP_{i} = 0 | X_{oi})}\right)$
	}
	\end{equation}
	\item Let $\nu_{ijk}$ the $(j,k)^{th}$ element of $\boldsymbol{V}_i^{-1}$. We need:
	\begin{itemize}
		\setlength\itemsep{0.5em}
		\item the weighted sum of $Y_{ij}$: $\sum_{j = 1}^{n_i}\sum_{i = 1}^{n_i}\textcolor<2>{red}{\nu_{ijk}}Y_{ij}$
		\item the weighted sum of $Y_{ij}t_{ik}$: $\sum_{j = 1}^{n_i}\sum_{i = 1}^{n_i}\textcolor<2>{red}{\nu_{ijk}}Y_{ij}t_{ik}$
		\item the weighted sum of $t_{ij}$: $\sum_{j = 1}^{n_i}\sum_{i = 1}^{n_i}\textcolor<2>{red}{\nu_{ijk}}t_{ij}$
		\item the weighted sum of $t_{ij}t_{ik}$: $\sum_{j = 1}^{n_i}\sum_{i = 1}^{n_i}\textcolor<2>{red}{\nu_{ijk}}t_{ij}t_{ik}$
		\item the confounder: $c_i$
		\item the interaction between the weighted sum of $t_{ij}$ and the confounder
		\item the weighted sum of the confounder: $\sum_{j = 1}^{n_i}\sum_{i = 1}^{n_i}\textcolor<2>{red}{\nu_{ijk}}c_i$
		\item the sum of all $\textcolor<2>{red}{\nu_{ijk}}$
	\end{itemize}
\end{itemize}
\end{frame}

\begin{frame}
\frametitle{Case 2: Balanced and Incomplete Data (cont'd)}
\begin{itemize}
	\item We estimate $\nu_{ijk}$ iteratively
	\item We estimate the initial set of weights $\nu_{ijk}$ from a model that does not include $SNP$ and $SNP \times time$
	\item We use MI to impute $SNP$
	\item We fit the linear mixed effect model of interest and estimate the new set of $\nu_{ijk}$
	\item In a scenario where subjects are observed 4, 5 or 6 times, coefficient's estimate and standard error:
	\begin{table}[H]
		\centering
		\resizebox{0.95\textwidth}{!}{
			\begin{tabular}{c|ccccc}
				\hline
				Sampling & $\beta_0$ & $\beta_t$ & $\beta_s$ & $\beta_{st}$ & $\beta_c$ \\
				\hline
				\emph{Random} & 75.00 (0.54) & -1.00 (0.06) & -0.50 (1.01) & -0.49 (0.14) & -2.00 (0.22) \\
				\emph{Intercept} & 75.00 (0.42) & -1.00 (0.06) & -0.50 (0.66) & -0.49 (0.15) & -2.01 (0.21) \\
				\emph{Slope} & 75.00 (0.53) & -1.00 (0.04) & -0.48 (1.02) & -0.49 (0.10) & -2.01 (0.21) \\
				\hline
				\textbf{Truth} & 75.00 & -1.00 & -0.50 & -0.50 & -2.00 \\
				\hline
			\end{tabular}
		}
	\end{table}
\end{itemize}
\end{frame}
%-------------------------------------------

\begin{frame}
\frametitle{Case 3: Unbalanced Data}
\begin{itemize}
	\item The same algorithm can be used for cases where subjects are observed at random time points
	\item For each subject $i$ we generate $n_i$ observations from $t_i \sim U(0, 10)$
	\begin{table}[H]
		\centering
		\resizebox{0.95\textwidth}{!}{
			\begin{tabular}{c|ccccc}
				
				\hline
				Sampling & $\beta_0$ & $\beta_t$ & $\beta_s$ & $\beta_{st}$ & $\beta_c$ \\
				\hline
				\emph{Random} & 74.99 (0.39) & -1.00 (0.06) & -0.49 (1.04) & -0.48 (0.14) & -2.01 (0.23) \\
				\emph{Intercept} & 74.99 (0.30) & -1.00 (0.05) & -0.48 (0.71) & -0.50 (0.14) & -2.01 (0.22) \\
				\emph{Slope} & 75.01 (0.40) & -1.00 (0.04) & -0.52 (1.07) & -0.50 (0.10) & -2.00 (0.23) \\
				\hline
				\textbf{Truth} & 75.00 & -1.00 & -0.50 & -0.50 & -2.00 \\
				\hline
			\end{tabular}
		}
	\end{table}
\end{itemize}
\end{frame}


\section{Summary}
\begin{frame}
\frametitle{Summary}
\begin{itemize}
		\setlength\itemsep{0.8em}
		\item When exposure ascertaiment cost limit the sample size, it is desirable to target a sample of informative subjects
		\item Analysis is usually done using complete data, or combining partial data and complete data using full likelihood approaches or multiple imputation
		\item We introduced an MI approach that is easy to implement and applicable to different sampling schemes
		\item We presented a series of simulation studies to look at the performance of the MI approach
\end{itemize}
\end{frame}

\begin{frame}
\frametitle{Reference}
\begin{itemize}
	\setlength\itemsep{0.8em}
	\item Schildcrout J, Rathouz P, Zelnick L, Garbett S, and Heagerty P. Biased Sampling Design To Improve Research Eefficiency: Factors Influencing Pulmonary Function Over Time In Children With Asthma. The Annals of Applied Statistics, 9(2):731–753, 2015.
	\item Schildcrout J, Haneuse S, Tao R, Zelnick L, Schisterman E, Garbett S, Mercaldo N, Rathouz P, and Heagerty P. Two-Phase, Generalized Case-Control Designs for the Study of Quantitative Longitudinal Outcomes. American Journal of Epidemiology. 2019. 
\end{itemize}
\end{frame}


\begin{frame}
\frametitle{}

\begin{center}
\LARGE{Thank you!}
\end{center}

\vspace{0.5cm}

\begin{itemize}
\item \normalsize{chiara.di.gravio@vanderbilt.edu}
\end{itemize}

\end{frame}


%-------------------------------------------


\end{document}

